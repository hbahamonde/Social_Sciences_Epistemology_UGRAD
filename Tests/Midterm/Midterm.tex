%This is a LaTeX template for homework assignments
\documentclass{article}
\usepackage{amsmath}
\usepackage{import} % To import email.

\usepackage{multido}
%\newcommand{\cmd}{-x-}
\newcommand{\Repeat}{\multido{\i=1+1}}


\usepackage{geometry}
\geometry{
  %body={6.5in, 8.5in},
  left=0.7in,
  right=0.7in,
  top=0.7in,
  bottom=1in
}


\begin{document}

{\centering\section*{Midterm\\POLC-2300 ``Ciencias Sociales''}}

{\vspace{.5cm}\raggedright{\bf Nombre}: \line(1,0){200}}. %you can change the length of the lines by changing the number in the curly brackets
{\vspace{.5cm}\hspace{4.5cm}\raggedright{\bf Fecha}: \line(1,0){100}}. %you can change the length of the lines by changing the number in the curly brackets


{\vspace{.5cm}\raggedright \bf Profesor}: Hector Bahamonde.\\
{\bf Clase}: {\input{/Users/hectorbahamonde/RU/Teaching/Social_Sciences_Epistemology_UGRAD/time_class_1.txt}\unskip}; {\input{/Users/hectorbahamonde/RU/Teaching/Social_Sciences_Epistemology_UGRAD/time_class_2.txt}\unskip}.\\
{\bf Sala}: C306.


\vspace{0.5cm}\subsection*{Instrucciones Generales}


El examen se realizar\'a el \underline{{\input{/Users/hectorbahamonde/RU/Teaching/Social_Sciences_Epistemology_UGRAD/date_midterm.txt}\unskip}}, {\input{/Users/hectorbahamonde/RU/Teaching/Social_Sciences_Epistemology_UGRAD/time_class_2.txt}\unskip}. Ex\'amenes que se entreguen despu\'es de esa hora, no ser\'an recibidos, y tendr\'an una nota 1. El examen tiene tres (3) preguntas tipo ensayo. El examen vale {\input{/Users/hectorbahamonde/RU/Teaching/Social_Sciences_Epistemology_UGRAD/percentage_midterm.txt}\unskip}\% de la nota final. No se permiten materiales de apoyo de ning\'un tipo. Entra toda la materia hasta la clase del 1 de Oct. Las preguntas han sido sacadas de las lecturas, documentales, discusiones en clase y ayudant\'ia. Bajo ninguna circunstancia se permitir\'an notas, libros, o pedazos de papel escritos, etc. El uso de aparatos electr\'onicos (laptops, relojes inteligentes, celulares, tablets, etc.) est'a estrictamente prohibido. Cualquier violaci\'on a estas reglas ser\'a sancionada con un 1 en la prueba. Escritura dif\'icil de leer quitar\'a puntos. Usa oraciones enteras: no se permite el uso de punt\'eos, o esquemas. Devuelve todas las hojas al final de la prueba. Puedes salir de la sala en cualquier minuto, pero despu\'es no ser\'a posible entrar a la sala nuevamente.

\subsection*{Ensayos}

Cada respuesta debe tener al menos dos p\'aginas, pero nunca m\'as de tres. Respuestas m\'as cortas que el l\'imite, obtendr\'an un 1 en ese ensayo. El m\'inimo oficial esta designado por un s\'imbolo de ``$\star$''. Por favor, responde la pregunta. Esto es, (1) no escribas ni cambies la pregunta en tu ensayo, (2) evita dar respuestas indirectas. Cualquier informaci\'on que est\'e m\'as all\'a del m\'aximo, no ser\'a le\'ida. En total, habr\'an tres ensayos. Si hay mas de tres ensayos, s\'olo se leer\'an los primeros tres.
\\

\line(1,0){510}


{\bf Por favor \underline{responde} las siguientes DOS preguntas.} 


\begin{enumerate}
    \item {\input{/Users/hectorbahamonde/RU/Teaching/Teaching_Questions_Database/Ciencia_Social_ESP/1_esp.txt}\unskip}
    
    \item {\input{/Users/hectorbahamonde/RU/Teaching/Teaching_Questions_Database/Ciencia_Social_ESP/2_esp.txt}\unskip}
\end{enumerate}    

\line(1,0){510}


{\bf Por favor \underline{escoge} UNA de las siguientes dos preguntas.} 


\begin{enumerate}
    \item {\input{/Users/hectorbahamonde/RU/Teaching/Teaching_Questions_Database/Ciencia_Social_ESP/3_esp.txt}\unskip} 
    
   \item {\input{/Users/hectorbahamonde/RU/Teaching/Teaching_Questions_Database/Ciencia_Social_ESP/4_esp.txt}\unskip} 
\end{enumerate}    

\line(1,0){510}


\clearpage
\newpage

% First Essay
\subsection*{Ensayo N\'umero \line(1,0){40}.}
\Repeat{24}{\line(1,0){510}\vspace{0.5cm}\\}
\Repeat{24}{\line(1,0){505}\vspace{0.5cm}\\}
\clearpage
\newpage
\Repeat{12}{\hspace{-5mm}\line(1,0){510}\vspace{0.5cm}\\}
{$\star$}\\
\Repeat{12}{\hspace{-5mm}\line(1,0){510}\vspace{0.5cm}\\}
\clearpage
\newpage

% Second Essay
\subsection*{Ensayo N\'umero \line(1,0){40}.}
\Repeat{24}{\line(1,0){510}\vspace{0.5cm}\\}
\clearpage
\newpage
\Repeat{12}{\hspace{-5mm}\line(1,0){510}\vspace{0.5cm}\\}
{$\star$}\\
\Repeat{12}{\hspace{-5mm}\line(1,0){510}\vspace{0.5cm}\\}
\clearpage
\newpage



% Organization 1
\subsection*{Esta secci\'on no sera le\'ida ni corregida. Esta secci\'on es para que tu organices tus respuestas (dibujando esquemas, por ej.), si fuera necesario.}
\clearpage
\newpage

% Organization 2
\subsection*{Esta secci\'on no sera le\'ida ni corregida. Esta secci\'on es para que tu organices tus respuestas (dibujando esquemas, por ej.), si fuera necesario.}
\clearpage
\newpage

% credible committments: AR (why they democratize instead of a one time transfer?)

\end{document}