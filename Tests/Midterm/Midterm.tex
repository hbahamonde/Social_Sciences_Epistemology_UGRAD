%This is a LaTeX template for homework assignments
\documentclass{article}
\usepackage{amsmath}
\usepackage{import} % To import email.

\usepackage{multido}
%\newcommand{\cmd}{-x-}
\newcommand{\Repeat}{\multido{\i=1+1}}


\usepackage{geometry}
\geometry{
  %body={6.5in, 8.5in},
  left=0.7in,
  right=0.7in,
  top=0.7in,
  bottom=1in
}


\begin{document}
\subsection*{Gira esta p\'agina s\'olo cuando el profesor lo indique.}

\clearpage
\newpage

{\centering\section*{Midterm\\POLC-2300 ``Ciencias Sociales''}}

{\vspace{.5cm}\raggedright{\bf Nombre}: \line(1,0){200}}. %you can change the length of the lines by changing the number in the curly brackets
{\vspace{.5cm}\hspace{4.5cm}\raggedright{\bf Fecha}: \line(1,0){100}}. %you can change the length of the lines by changing the number in the curly brackets


{\vspace{.5cm}\raggedright \bf Profesor}: H\'ector Bahamonde, PhD.\\
{\bf Clase}: {\input{/Users/hectorbahamonde/RU/Teaching/Social_Sciences_Epistemology_UGRAD/time_class_1.txt}\unskip}; {\input{/Users/hectorbahamonde/RU/Teaching/Social_Sciences_Epistemology_UGRAD/time_class_2.txt}\unskip}.\\
{\bf Sala}: C306.


\vspace{0.5cm}\subsection*{Instrucciones Generales}


El examen se realizar\'a el \underline{{\input{/Users/hectorbahamonde/RU/Teaching/Social_Sciences_Epistemology_UGRAD/date_midterm.txt}\unskip}}, {\input{/Users/hectorbahamonde/RU/Teaching/Social_Sciences_Epistemology_UGRAD/time_class_2.txt}\unskip}. Ex\'amenes que se entreguen despu\'es de esta hora, no ser\'an recibidos, y tendr\'an una nota 1. El examen tiene {\input{/Users/hectorbahamonde/RU/Teaching/Social_Sciences_Epistemology_UGRAD/num_ensayos_midterm.txt}\unskip} preguntas tipo ensayo. El examen vale {\input{/Users/hectorbahamonde/RU/Teaching/Social_Sciences_Epistemology_UGRAD/percentage_midterm.txt}\unskip}\% de la nota final, y toda la materia hasta el {\input{/Users/hectorbahamonde/RU/Teaching/Social_Sciences_Epistemology_UGRAD/final_class_before_midterm.txt}\unskip} podr\'ia ser preguntada. Las preguntas han sido sacadas de las lecturas, documentales, discusiones en clase y ayudant\'ia. No se permiten materiales de apoyo de ning\'un tipo. Bajo ninguna circunstancia se permitir\'an apuntes, libros, o pedazos de papel escritos, etc. El uso de aparatos electr\'onicos (laptops, relojes inteligentes, celulares, tablets, etc.) est\'a estrictamente prohibido. Si necesitas usar el servicio higi\'enico, tendr\'as que dejar tu celular con el profesor. Cualquier violaci\'on a estas reglas ser\'a sancionada con un 1 en la prueba.  Escritura dif\'icil de leer quitar\'a puntos. Usa oraciones enteras: no se permite el uso de punt\'eos, o esquemas. Devuelve todas las hojas al final de la prueba. Puedes salir de la sala en cualquier minuto, pero despu\'es no ser\'a posible entrar a la sala nuevamente. 


 

\subsection*{Ensayos}

Cada ensayo debe tener {\bf al menos 1.5}, pero {\bf nunca m\'as de 3 p\'aginas}. Respuestas m\'as cortas que el l\'imite m\'inimo, obtendr\'an un 1 en ese ensayo. El m\'inimo oficial est\'a denotado por un s\'imbolo ``$\star$''. Por favor, responde la pregunta. Esto es, (1) no reescribas, replantees, ni cambies la pregunta en tu ensayo, (2) evita dar respuestas indirectas. Cualquier informaci\'on que est\'e m\'as all\'a del m\'aximo, no ser\'a le\'ida. En total, habr\'an {\input{/Users/hectorbahamonde/RU/Teaching/Social_Sciences_Epistemology_UGRAD/num_ensayos_midterm.txt}\unskip} ensayos. Si hay m\'as de {\input{/Users/hectorbahamonde/RU/Teaching/Social_Sciences_Epistemology_UGRAD/num_ensayos_midterm.txt}\unskip} ensayos, s\'olo se leer\'an los primeros {\input{/Users/hectorbahamonde/RU/Teaching/Social_Sciences_Epistemology_UGRAD/num_ensayos_midterm.txt}\unskip}. Las respuestas no necesariamente deben estar en orden correlativo.
\\

\line(1,0){510}


{\bf Por favor \underline{responde} las siguientes DOS preguntas:} 


\begin{enumerate}
    \item {\input{/Users/hectorbahamonde/RU/Teaching/Teaching_Questions_Database/Ciencia_Social_ESP/1_esp.txt}\unskip}
    
    \item {\input{/Users/hectorbahamonde/RU/Teaching/Teaching_Questions_Database/Ciencia_Social_ESP/2_esp.txt}\unskip}
\end{enumerate}    

\line(1,0){510}


{\bf Por favor \underline{escoge} UNA de las siguientes dos preguntas:} 


\begin{enumerate}
    \item {\input{/Users/hectorbahamonde/RU/Teaching/Teaching_Questions_Database/Ciencia_Social_ESP/3_esp.txt}\unskip} 
    
   \item {\input{/Users/hectorbahamonde/RU/Teaching/Teaching_Questions_Database/Ciencia_Social_ESP/4_esp.txt}\unskip} 
\end{enumerate}    

\line(1,0){510}


\clearpage
\newpage

% First Essay
\subsection*{Ensayo N\'umero \line(1,0){40}.}
\Repeat{24}{\line(1,0){510}\vspace{0.5cm}\\}
\clearpage
\newpage
\Repeat{12}{\hspace{-5mm}\line(1,0){510}\vspace{0.5cm}\\}
{$\star$}\\
\Repeat{12}{\hspace{-5mm}\line(1,0){510}\vspace{0.5cm}\\}
\Repeat{24}{\line(1,0){510}\vspace{0.5cm}\\}
\clearpage
\newpage

% Second Essay
\subsection*{Ensayo N\'umero \line(1,0){40}.}
\Repeat{24}{\line(1,0){510}\vspace{0.5cm}\\}
\clearpage
\newpage
\Repeat{12}{\hspace{-5mm}\line(1,0){510}\vspace{0.5cm}\\}
{$\star$}\\
\Repeat{12}{\hspace{-5mm}\line(1,0){510}\vspace{0.5cm}\\}
\Repeat{24}{\line(1,0){510}\vspace{0.5cm}\\}
\clearpage
\newpage


% Third Essay
\subsection*{Ensayo N\'umero \line(1,0){40}.}
\Repeat{24}{\line(1,0){510}\vspace{0.5cm}\\}
\clearpage
\newpage
\Repeat{12}{\hspace{-5mm}\line(1,0){510}\vspace{0.5cm}\\}
{$\star$}\\
\Repeat{12}{\hspace{-5mm}\line(1,0){510}\vspace{0.5cm}\\}
\Repeat{24}{\line(1,0){510}\vspace{0.5cm}\\}
\clearpage
\newpage


% Organization 1
\subsection*{Esta secci\'on no sera le\'ida ni corregida. Esta secci\'on es para que tu organices tus respuestas (dibujando esquemas, por ej.), si fuera necesario.}
\clearpage
\newpage

% Organization 2
\subsection*{Esta secci\'on no sera le\'ida ni corregida. Esta secci\'on es para que tu organices tus respuestas (dibujando esquemas, por ej.), si fuera necesario.}
\clearpage
\newpage

% credible committments: AR (why they democratize instead of a one time transfer?)

\end{document}